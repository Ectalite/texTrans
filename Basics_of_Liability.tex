% !TeX spellcheck = en_US
\section{Liability (Haftung)}
\subsection{Damage}
\begin{compactitem}
	\item Damage $\neq$ Satisfaction
	\item Damage = Financial compensation to refund an asset, that has been reduced (or couldn’t rise) by an unlawfully action.
	\item The exact amount has to be proven by submitting invoices, accounting documents etc. The judge has the right to estimate the damage. But he needs evidences.
	\item Satisfaction = Financial compensation for physical or mental injury upon discretion.
\end{compactitem}

\subsection{Multiple \& cumulative requirements for compensation}
\begin{compactenum}
	\item Damage
	\item Unlawfulness
	\item Adequate causal connection
	\item Fault
	\item (no expiry of liability claims)
\end{compactenum}

\subsection{Unlawfulness}
\begin{compactitem}
	\item Includes the proof that a legally protected right has been illegally infringed.
	\item Any infringement of an absolutely protected legal right such as the right to life and limb and property is per se illegal.
	\item Infringement of a relative protected legal right, such as an asset, is only illegal if a standard of protection has been violated.
	\item As an exception and if there are special reasons, the illegality can be suspended. Possible justification grounds include self-defense, consent by the injured party, official obligation etc.
\end{compactitem}

\subsection{Adequate causal connection}
\begin{compactitem}
	\item The proof must be provided that there is a direct causal connection that makes sense in the normal course of life between the damaging event and the damages being claimed.
	\item This means that in the normal course of things and according to general life experience, the damaging	behavior was liable to cause damages of the type that occurred (Definition by the Swiss Federal Court).
\end{compactitem}

\subsection{Fault}
\begin{compactitem}
	\item From an objective viewpoint, fault requires that the damaging party can be judged. From a subjective viewpoint, it requires intent or negligence (Vorsatz oder Fahrlässigkeit). Even minor negligence, ie. simply a violation of due care, is sufficient to constitute fault.
	\item Fault does not need to be proven in the case of contractual liability since it is assumed by law! (see also Art. 97 OR).
	\item Strict liability means that the damaging party is liable irrespective of fault. Liability is founded here on the fact that he is responsible eg. for a generally dangerous circumstance. For example, in animal keeper's liability according to OR 56, proof that the damages were caused by an animal under the care of the animal keeper and that an adequate causal relationship exists is sufficient.
\end{compactitem}

\subsection{Expiry of liability claims}
\begin{compactitem}
	\item Liability claims are subject to a statute of limitations. When the statute of limitations expires, the claims do not go away; they simply can no longer be legally executed against the will of the liable party.
	\item The general statutes of limitations in liability law are:
	\begin{compactitem}
		\item 1 year relative/10 years absolute in non-contractual liability (Art. 60 OR)
		\item 5 or 10 years in contractual liability (Art. 127 ff. OR)
		\item other rules can be found in some special laws
	\end{compactitem}
\end{compactitem}

\subsection{Contractual and non contractual liability}
\begin{compactitem}
	\item In liability law, a distinction is made between contractual and non-contractual liability. A contractual liability is in effect when an existing contract between the damaging party and the injured party has been violated.
	\item A non-contractual liability is in effect when a person cause a damage, regardless of whether a contract is in place or not.
\end{compactitem}

\subsection{Fault-dependent and fault-independent (causal) liability}
\begin{compactitem}
	\item With regard to non-contractual liabilities, the legislature distinguishes between general fault-dependent liability situations (fault-based liability Art. 41, CO) and fault-independent liability situations listed to some degree in the OR, CC and special laws (causal liability, e.g. in Art. 55, OR, Art. 333, CC).
	\item Product liability is one of the causal independent liability situations and is regulated in a special law, the Product Liability Law (PrHG).
\end{compactitem}

\subsection{Product liability}
\begin{compactitem}
	\item WHAT? Only personal injuries or property damages of consumers are covered (Art. 1, PrHG).
	\item WHO? The manufacturer or the importer are liable (Art. 2, PrHG).
	\item WHAT? Products are movable things (Art. 3 PrHG). No software!
	\item EXCEPTIONS? Manufacturer are not liable eg. if they can prove that they did not bring the product to market or the defect that caused the damage did not exist at the time the product was delivered (Art. 5 PrHG)
	\item WHEN? Claims according to PrHG expire 3 years after the date on which the damaged party becomes aware or should have become aware of the damages, the defect and the manufacturer (Art. 9, PrHG).
\end{compactitem}

\subsection{Splitting the damage between several responsible}
\begin{compactitem}
	\item What, when multiple people caused a damage? Then they are regarded as a „Simple Society“.
	\item Externally (damaged party vs. group) they are liable in solidarity („one for all, all for one“)!
	\item Internally (any member of the group) they are liable equally.
\end{compactitem}

\subsection{Solidarity}
\begin{compactitem}
	\item If multiple persons are causally liable, they are mutually liable in solidarity (Art.	143 ff. OR).
	\item The damaged party can demand either parts of the damages from all parties or the entire amount from one liable party.
	\item Internally, the individual liable parties have recourse options between themselves.
	\item Be careful! Simple Partnership (Art. 530 ff OR)!
\end{compactitem}

\subsection{Regression/Recourse}
Regress/recourse is the right of ie. an insurance company to demand from a responsible (customer) of indemnification (Entschädigung) of damages.

\subsection{Several limits of transferring liability}
\begin{compactitem}
	\item Company-structure (corporate body like AG, GmbH)
	\item Generally no „reach through“ the company to the individuals, but... (ie. an employee or the management might be responsible)
	\item Liability of employers (Art. 55 CO). Principle of „cura in eligendo, instruendo, custodiendo e organisando“
	\item Contractual limitations (but Art. 100 I OR)
\end{compactitem}

\subsection{Gross negligence (Grobe Fahrlässigkeit)}
Gross negligence is a conscious and voluntary disregard of the need to use reasonable care, which is likely to cause foreseeable grave injury or harm to persons, property, or both. It is conduct that is extreme when compared with ordinary negligence, which is a mere failure to exercise reasonable care. Ordinary negligence and gross negligence differ in degree of inattention, while both differ from willful and wanton conduct, which is conduct that is reasonably considered to cause injury.

\subsection{Some hints to civil procedure law}
\begin{compactitem}
	\item Main question: who is entitled to claim a (law)suit on which court (territorial/subject matter) under what law
	\item Generally: Plaintiff against Defendant
	\item Art. 8 CC - rule of evidence: Unless the law provides	otherwise, the burden of proving the existence of an alleged fact shall rest on the person who derives rights from that fact.
	\item Principle: Litigation shall be preceded by an attempt at conciliation (Schlichtung) before a conciliation authority. (Art. 197 CPL))
\end{compactitem}

\subsection{Before you start a lawsuit}
\begin{compactitem}
	\item Check the deadlines!
	\item Check your (financial) risks... If you lose the lawsuit you have to pay the court fees, your lawyer and the defendant lawyer... (or you have an insurance or no asset)
	\item Check your evidences...
\end{compactitem}

\subsection{Legal base}
\subsubsection{Obligation in tort - ART. 41 OR}
Any person who unlawfully causes loss or damage to another, whether willfully or negligently, is obliged to provide compensation.

\textit{Wer einem andern widerrechtlich Schaden zufügt, sei es mit Absicht, sei es aus Fahrlässigkeit, wird ihm zum Ersatze verpflichtet.}

\subsubsection{Solidarity - ART. 144 OR}
\begin{compactenum}
	\item A creditor may at his discretion request partial performance of the obligation from each joint and several debtor or else full performance from any one of them.
	\item All the debtors remain under the obligation until the entire claim has been redeemed.
\end{compactenum}

\textit{\begin{compactenum}
	\item Der Gläubiger kann nach seiner Wahl von allen Solidarschuldnern je nur einen Teil oder das Ganze fordern.
	\item Sämtliche Schuldner bleiben so lange verpflichtet, bis die ganze Forderung getilgt ist.
\end{compactenum}}

\subsubsection{Liability in simple partnership - ART. 530 OR}
A partnership is a contractual relationship in which two or more persons agree to combine their efforts or resources in order to achieve a common goal.

\textit{Gesellschaft ist die vertragsmässige Verbindung von zwei oder mehreren Personen zur Erreichung eines gemeinsamen Zweckes mit gemeinsamen Kräften oder Mitteln.}

\subsubsection{Liability in simple partnership - ART. 533 OR}
Unless otherwise agreed, each partner has an equal share in profits and losses regardless of the nature and amount of his contribution.

\textit{Wird es nicht anders vereinbart, so hat jeder Gesellschafter, ohne Rücksicht auf die Art und Grösse seines Beitrages, gleichen Anteil an Gewinn und Verlust.}

\subsubsection{Liability in simple partnership - ART. 537 OR}
Where one partner incurs expenses or contracts liabilities in connection with affairs conducted on behalf of the partnership or suffers losses as a direct consequence of his management activities or the intrinsically associated risks, the other partners share his liability.

\textit{Für Auslagen oder Verbindlichkeiten, die ein Gesellschafter in den Angelegenheiten der Gesellschaft macht oder eingeht, sowie für Verluste, die er unmittelbar durch seine Geschäftsführung oder aus den untrennbar damit verbundenen Gefahren erleidet, sind ihm die übrigen Gesellschafter haftbar.}

\subsubsection{Liability of employers - ART. 55 OR}
\begin{compactenum}
	\item An employer is liable for the loss or damage caused by his employees or ancillary staff in the performance of their work unless the proves that he took all due care to avoid a loss or damage of this type or that the loss or damage would have occured even if all due care had been taken.
	\item The employer has a right of recourse against the person who caused the loss or damage to the extent that such person is liable in damages.
\end{compactenum}

\textit{\begin{compactenum}
	\item Der Geschäftsherr haftet für den Schaden, den seine Arbeitnehmer oder andere Hilfspersonen in Ausübung ihrer dienstlichen oder geschäftlichen Verrichtungen verursacht haben, wenn er nicht nachweist, dass er alle nach den Umständen gebotene Sorgfalt angewendet hat, um einen Schaden dieser Art zu verhüten, oder dass der Schaden auch bei Anwendung dieser Sorgfalt eingetreten wäre.
	\item Der Geschäftsherr kann auf denjenigen, der den Schaden gestiftet hat, insoweit Rückgriff nehmen, als dieser selbst schadenersatzpflichtig ist.
\end{compactenum}}

\subsubsection{Liability of employers - ART. 321e OR}
\begin{compactenum}
	\item The employee is liable for any loss or damage he causes to the employer whether willfully or by negligence.
	\item The extent of duty of care owed by the employee is determined by the individual employment contract, taking due account of the occupational risk, level of training and technical knowledge associated with the work as well the employee’s aptitudes and skills of which the employer was or should have been aware.
\end{compactenum}

\textit{\begin{compactenum}
	\item Der Arbeitnehmer ist für den Schaden verantwortlich, den er absichtlich oder fahrlässig dem Arbeitgeber zufügt.
	\item Das Mass der Sorgfalt, für die der Arbeitnehmer einzustehen hat, bestimmt sich nach dem einzelnen Arbeitsverhältnis, unter Berücksichtigung des Berufsrisikos, des Bildungsgrades oder der Fachkenntnisse, die zu der Arbeit verlangt werden, sowie der Fähigkeiten und Eigenschaften des Arbeitnehmers, die der Arbeitgeber gekannt hat oder hätte kennen sollen.
\end{compactenum}}

\subsubsection{Limitation of liability - ART. 100 OR}
\begin{compactenum}
	\item Any agreement purporting to exclude liability for unlawful intent or gross negligence in advance is void.
\end{compactenum}

\textit{\begin{compactenum}
	\item Eine zum voraus getroffene Verabredung, wonach die Haftung für rechtswidrige Absicht oder grobe Fahrlässigkeit ausgeschlossen sein würde, ist nichtig.
\end{compactenum}}

\subsubsection{Actions in contract - ART. 31 CPC (ZPO)}
The court at the domicile or registered office of the defendant or at the place where the characteristic performance must be rendered has jurisdiction over actions related to contracts.

\textit{Für Klagen aus Vertrag ist das Gericht am Wohnsitz oder Sitz der beklagten Partei oder an dem Ort zuständig, an dem die charakteristische Leistung zu erbringen ist.}

\subsubsection{Employment law - ART. 34 CPC}
The court at the domicile or registered office of the defendant or where the employee normally carries out his or her work has jurisdiction to decide actions relating to employment law.

\textit{Für arbeitsrechtliche Klagen ist das Gericht am Wohnsitz oder Sitz der beklagten Partei oder an dem Ort, an dem die Arbeitnehmerin oder der Arbeitnehmer gewöhnlich die Arbeit verrichtet, zuständig.}